\section{Заключение}
В рамках проведенной работы были уточнены детали устройства метода рендеринга с использованием кадровых графов, написана реализация кадрового графа с поддержкой нескольких командных очередей, предложен ряд новых архитектурных решений в проектировании Render Hardware Interface, а также написан рендерер, использующий современные подходы к графике, такие как вычислительные шейдеры и gpu-driven rendering, в качестве доказательства валидности нового дизайна.

В силу комплексности современных графических API, при дальнейших исследованиях планируется повышать производительность и качество алгоритма планировки задач кадрового графа. Планируется реализовать оптимизированную имплементацию компиляции в случае использования только одной или двух очередей, поскольку в таком случае сложность сильно понижается и нет необходимости рассматривать некоторые сценарии, ввиду их невозможности.

Еще одним крупным направлением для дальнейших исследований является написание фронтенда декларации графов в данных. Это может быть как интеграция в систему шейдеров (как это сделано в \cite{amd_rps_sdk}), так и построение с помощью визуального редактора узлов, подобно современным редакторам игровых скриптов или материалов \cite{material_editor_ue5}.

Подход интеграции кадровых графов в RHI весьма перспективен в применении к современным приложениям реального времени, поскольку на данном низком уровне абстракции держится весь высокоуровневый рендеринг, который становится все более комплексным с каждым годом. Данный подход позволит более эффективно использовать ресурсы GPU, позволяя графическим разработчикам концентрироваться на совершенствовании отдельных графических эффектов, делегируя связи и планировку кадровым графам, а поэтому данный подход заслуживает дальнейшего исследования и усовершенствования.