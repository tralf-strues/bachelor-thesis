\usepackage{fontspec}
\usepackage{polyglossia}
\usepackage[a4paper, lmargin=30mm, rmargin=15mm, tmargin=20mm, bmargin=20mm]{geometry}
\usepackage{multirow}

\setdefaultlanguage{russian}
\setotherlanguage{english}

\defaultfontfeatures{Ligatures=TeX}

\setmainfont{Times New Roman}
\setmonofont{Courier New}
\setsansfont{Arial}

\newfontfamily\cyrillicfont{Times New Roman}
\newfontfamily\cyrillicfontsf{Arial}
\newfontfamily\cyrillicfonttt{Courier New}

\newfontfamily\englishfont{Times New Roman}
\newfontfamily\englishfontsf{Arial}
\newfontfamily\englishfonttt{Courier New}

\linespread{1.5}

\usepackage[backend=biber,
  bibencoding=utf8,
  sorting=none,
  style=gost-numeric,
  language=autobib,
  autolang=other,
  clearlang=true,
  defernumbers=true,
  sortcites=true,
  doi=true,
  isbn=true,
  ]{biblatex}

\usepackage[dvipsnames]{xcolor}

\definecolor{coolgrey}{rgb}{0.55, 0.57, 0.67} % This is the color of character in diagrams

\usepackage[
  hidelinks,
  colorlinks=true,
  linkcolor=Blue, % This is the color of table of contents and figure references
  citecolor=Blue, % This is the color of citations
  urlcolor=Blue % This is the colro of urls
]{hyperref}
\renewcommand{\UrlFont}{\small\rmfamily\tt}

\renewcommand\thesection{\arabic{section}}

\usepackage{amsfonts}
\usepackage{amsmath}

\usepackage{tikz} % This allows to make diagrams
\usetikzlibrary{arrows}
\usetikzlibrary{arrows.meta}
\usetikzlibrary{positioning}

\usepackage{caption}
\captionsetup[figure]{font=footnotesize}
\captionsetup[lstlisting]{font=footnotesize}

\usepackage{graphicx}
\graphicspath{ {./images/} } % This sets path to directory containing images.
\addto\captionsrussian{\renewcommand{\figurename}{Рисунок}} % This changes caption from 'Рис.' to 'Рисунок'.
\addto\captionsrussian{\renewcommand{\lstlistingname}{Листинг}} % This changes caption from 'Listing.' to 'Листинг'.

\usepackage{wrapfig}
\usepackage{subfigure}
\usepackage{subcaption}

\usepackage{titlesec}
% \setcounter{secnumdepth}{4}
% \setcounter{tocdepth}{4}

\usepackage{listings}
\usepackage{xcolor}
\usepackage{verbatimbox}

\definecolor{codegreen}{rgb}{0,0.6,0}
\definecolor{codegray}{rgb}{0.6,0.6,0.6}
\definecolor{codepurple}{rgb}{0.58,0,0.82}
\definecolor{backcolour}{rgb}{0.95,0.95,0.95}

\lstdefinestyle{cppstyle}{
  % backgroundcolor=\color{backcolour},
  commentstyle=\color{codegreen},
  keywordstyle=\color{magenta},
  numberstyle=\tiny\color{codegray},
  stringstyle=\color{codepurple},
  basicstyle=\ttfamily\scriptsize,
  % basicstyle=\scriptsize,
  breakatwhitespace=false,
  breaklines=true,
  captionpos=b,
  keepspaces=true,
  numbers=left,
  numbersep=5pt,
  showspaces=false,
  showstringspaces=false,
  showtabs=false,
  tabsize=2,
  % frame=tB
}

\lstdefinestyle{pseudocodestyle}{
  mathescape=true,
  frame=tB,
  numbers=left, 
  % numberstyle=\tiny,
  numberstyle=\tiny\color{codegray},
  numbersep=5pt,
  basicstyle=\scriptsize,
  keywordstyle=\color{black}\bfseries,
  keywords={,return, function, in, if, else, while, true, false, and, not} %add the keywords you want, or load a language as Rubens explains in his comment above.
  numbers=left,
  captionpos=b,
  breakatwhitespace=false
}

% \lstset{style=mystyle}

\usepackage{regexpatch}% http://ctan.org/pkg/regexpatch
\usepackage{listings}% http://ctan.org/pkg/listings
\makeatletter

\newcommand{\lstlistcplusplusname}{List of C++}
\lst@UserCommand\lstlistofcplusplus{\bgroup
    \let\contentsname\lstlistcplusplusname
    \let\lst@temp\@starttoc \def\@starttoc##1{\lst@temp{loc}}%
    \tableofcontents \egroup}
\lstnewenvironment{cpp}[1][]{%
  \renewcommand{\lstlistingname}{C++ Листинг}%
  \xpatchcmd*{\lst@MakeCaption}{lol}{loc}{}{}%
  \lstset{style=cppstyle, language=C++,#1}}
  {}

\newcommand{\lstlistpseudocodename}{List of Pseudocode}
\lst@UserCommand\lstlistofpseudocode{\bgroup
    \let\contentsname\lstlistpseudocodename
    \let\lst@temp\@starttoc \def\@starttoc##1{\lst@temp{lop}}%
    \tableofcontents \egroup}
\lstnewenvironment{pseudocode}[2][]{%
  \renewcommand{\lstlistingname}{Алгоритм}%
  \xpatchcmd*{\lst@MakeCaption}{lol}{lop}{}{}%
  \lstset{style=pseudocodestyle,#1}}
  {}
\makeatother
